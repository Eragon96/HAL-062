\documentclass{article}
\usepackage[T1]{fontenc}
\usepackage[polish]{babel}
\usepackage[utf8]{inputenc}
\usepackage{lmodern}
\usepackage{textcomp}
\selectlanguage{polish}
\author{Konrad Aleksiejuk}
\title{Dokumentacja oprogramowania do projektu naukowego HAL-062}

\begin{document}
\maketitle
\tableofcontents

\newpage
\section{Ramki komunikacyjne}
\subsection{UART}
Komunikacja UART wykorzystywana jest do przesyłania danych pomiędzy sterownikiem głównym (Nucleo F446) a urządzeniami odpowiedzialnymi za łączność bezprzewodową (Router) oraz do połączenia między sterownikiem głównym a komputerem pokładowym (Jetson).

W celu zapewnienia płynnej i niezawodnej komunikacji między operatorem a łazikiem dane są przesyłane w postaci ramek o zdefiniowanym formacie:
\begin{itemize}
\item znak ASCII '\#' - znak początku transmisji,
\item identyfikator ramki - liczba (w zakresie 0-255) określająca typ ramki, czyli jej długość i typ przesyłanych danych,
\item bity danych - dane w różnej postaci zależnie od typu ramki.
\end{itemize}

Powyższa konstrukcja podyktowana jest koniecznością przesyłania w obu kierunkach różnorodnych danych i parametrów, dlatego ramka musi mieć możliwość przyjmowania różnej długości. Ze względu na możliwość przyjmowania przez przesyłane dane dowolnej wartości (0-255) niemożliwe jest wykorzystanie znaku końca transmisji, gdyż istnieje ryzyko wystąpienia takiego znaku wśród przesyłanych danych. 

\subsection{CAN}
CAN jest szeregową magistralą komunikacyjną powszechnie wykorzystywaną w przemyśle i motoryzacji. Charakteryzuje się dużą odpornością na zakłócenia i łatwością obsługi. W łaziku została wykorzystana do komunikacji pomiędzy wszystkimi modułami znajdującymi się na pokładzie łazika. Transmisja może odbywać się z prędkością do 1Mb/s. Medium pośredniczącym w transmisji jest skrętka dwu-przewodowa, w której przesyłany jest napięciowy sygnał różnicowy. 

Mikrokontrolery STM32 wykorzystywane na łaziku wspierają protokół CAN w wersji 2.0A oraz 2.0B. Na potrzeby projektu zastosowano wersję 2.0A, gdyż przestrzeń adresowa dostarczana przez tę wersję jest zdecydowanie wystarczająca. Magistrala działa w konfiguracji multi-master. W praktyce oznacza to, że każda z płytek nadaje informację na magistralę niezależnie, ze stałym interwałem czasowym. W większości przypadków dane przesyłane za pośrednictwem magistrali nie są danymi kluczowymi dla działania innych podsystemów, a mają jedynie formę informacyjną dla operatora. Dodatkowo częstotliwość nadawania ramek na magistralę nie przekracza 50Hz, co przy szybkości transmisji rzędu 0.5Mb/s nie wprowadza ryzyka zablokowania magistrali i utraty części danych.

Ramki identyfikowane są na podstawie adresu ramki (StdID). \\
\textbf{Adres StdID jest równoznaczny z identyfikatorem ramki w komunikacji UART.} Przyjmują one takie same wartości i odpowiadają tym samym długościom ramek. 

Zakresy adresów są przydzielane do konkretnych podsystemów:
\begin{itemize}
\item 100-159 platforma jezdna
\item 160-199 manipulator
\item 200-239 laboratorium
\end{itemize}

\subsection{Zdefiniowane ramki}
\subsubsection*{100-159 platforma jezdna}
	\begin{enumerate}
	\setcounter{enumi}{99}
	\item Ramka do sterownika, określa prędkość i kierunek ruchu łazika, odpowiada odczytom ze standardowego joysticka. Długość ramki: 2 znaki.
		\begin{itemize}
		\item \arabic{enumi}
		\item wskazanieY wychylenie drążka w kierunku Y (prawo lewo) 
		\item wskazanieX wychylenie drążka w kierunku X (przód tył)
		\end{itemize}
	 Uwagi: Wskazania joysticku są odwrócone względem intuicyjnego postrzegania (maksymalne wychylenie do przodu odpowiada wartości -100), 	wartość 0,0 oznacza położenie centralne, a tym samym łazik pozostaje bez ruchu. 
	\item Ramka do sterownika, zezwolenie na uruchomienie silników. Bez wysłania tej ramki silniki nie reagują na polecenie 100 (nastawy prędkości). Długość ramki: 1 znak.
		\begin{itemize}
		\item \arabic{enumi}
		\item '1' wartość char (0x31) oznacza zezwolenie na ruch każda inna wartość oznacza brak zezwolenia i powoduje zatrzymanie napędów.
		\end{itemize}
  	\item Ramka zawiera nastawy prędkości silników po prawej stronie
  		\begin{itemize}
  		\item \arabic{enumi}
 		\item Data[0] = prędkość silnika pierwszego 
 		\item Data[1] = prędkość silnika drugiego
  		\item Data[2] = prędkość silnika trzeciego
 		\end{itemize}
 	Uwagi: Wartości prędkości to int8 w zakresie od -100 do 100 gdzie 0 oznacza zatrzymanie, 100 maksymalną prędkość obrotów do przodu, a -100 maksymalną prędkość obrotów do tyłu.
  	\item Ramka zawiera nastawy prędkości silników po lewej stronie
  		\begin{itemize}
  		\item \arabic{enumi}
  		\item Data[0] = prędkość silnika pierwszego 
  		\item Data[1] = prędkość silnika drugiego
  		\item Data[2] = prędkość silnika trzeciego
  		\end{itemize}
   	Uwagi: Wartości prędkości to int8 w zakresie od -100 do 100 gdzie 0 oznacza zatrzymanie, 100 maksymalną prędkość obrotów do przodu, a -100 maksymalną prędkość obrotów do tyłu
	\item Ramka do operatora, zawiera dane o prądzie pobieranym przez silniki po prawej stronie. Długość ramki: 6 znaków.
 		\begin{itemize}
 		\item \arabic{enumi}
 	 	\item currentValue1 \& 0xFF;
  		\item (currentValue1 \& 0xFF00)>>8 ;
  		\item currentValue2 \& 0xFF;
  		\item (currentValue2 \& 0xFF00)>>8 ;
  		\item currentValue3 \& 0xFF;
  		\item (currentValue3 \& 0xFF00)>>8 ;
  		\end{itemize} 
  	Uwagi: currentValue to int16 zawierający prąd pobierany przez kolejne silniki, wartość currentValue odpowiada odczytom z przetwornika ADC i należy ją przeliczyć na wartość pobieranego prądu. 
  	\item Ramka do operatora, zawiera dane o prądzie pobieranym przez silniki po lewej stronie. Długość ramki: 6 znaków.
 		\begin{itemize}
 		\item \arabic{enumi}
  		\item currentValue1 \& 0xFF;
  		\item (currentValue1 \& 0xFF00)>>8 ;
  		\item currentValue2 \& 0xFF;
  		\item (currentValue2 \& 0xFF00)>>8 ;
  		\item currentValue3 \& 0xFF;
  		\item (currentValue3 \& 0xFF00)>>8 ;
  		\end{itemize} 
  	Uwagi: currentValue to int16 zawierający prąd pobierany przez kolejne silniki, wartość currentValue odpowiada odczytom z przetwornika ADC i należy ją przeliczyć na wartość pobieranego prądu. 
  	\item Ramka do operatora, zawiera odczyty prędkości kół z enkoderów dla napędów z prawej strony. Długość ramki: 6 znaków. 
   	 	\begin{itemize}
    	\item \arabic{enumi}
  		\item enkPrędkość1 \& 0xFF;
  		\item (enkPrędkość1 \& 0xFF00)>>8 ;
  		\item enkPrędkość2 \& 0xFF;
  		\item (enkPrędkość2 \& 0xFF00)>>8 ;
  		\item enkPrędkość3 \& 0xFF;
  		\item (enkPrędkość3 \& 0xFF00)>>8 ;
  		\end{itemize}
    Uwagi: enkPrędkość to int16 zawierający prędkości obrotowe kolejnych kół.
    \item Ramka do operatora, zawiera odczyty prędkości kół z enkoderów dla napędów z lewej strony. Długość ramki: 6 znaków.
    	\begin{itemize}
    	\item \arabic{enumi}
  		\item enkPrędkość1 \& 0xFF;
  		\item (enkPrędkość1 \& 0xFF00)>>8 ;
  		\item enkPrędkość2 \& 0xFF;
  		\item (enkPrędkość2 \& 0xFF00)>>8 ;
  		\item enkPrędkość3 \& 0xFF;
  		\item (enkPrędkość3 \& 0xFF00)>>8 ;
  		\end{itemize}
    Uwagi: enkPrędkość to int16 zawierający prędkości obrotowe kolejnych kół.
    \item Ramka do operatora, zawiera pomiary napięcia na kolejnych celach baterii. Długość ramki: 8 znaków.
    	\begin{itemize}
    	\item \arabic{enumi}
    	\item Vbat1 \& 0xFF;
  		\item (Vbat1 \& 0xFF00)>>8 ;
  		\item Vbat2 \& 0xFF;
  		\item (Vbat2 \& 0xFF00)>>8 ;
  		\item Vbat3 \& 0xFF;
  		\item (Vbat3 \& 0xFF00)>>8 ;
  		\item Vbat4 \& 0xFF;
  		\item (Vbat4 \& 0xFF00)>>8 ;
    	\end{itemize}
    Uwagi: Vbat to napięcia zmierzone na kolejnych celach baterii. Są to wartości 12 bitowe, wymagają przeliczenia na wartości napięcia z uwzględnieniem dzielników napięcia zastosowanych na każdym z wejść analogowych (inne dla każdej z cel) 
   	\item Ramka zawiera odczyty temperatur baterii zmierzone przez BMS. Długość ramki: 6 znaków
  		\begin{itemize}
  		\item \arabic{enumi}
  		\item batTemp1
  		\item batTemp2
  		\item batTemp3
  		\item batTemp4
  		\item batTemp5
  		\item batTemp6	
  		\end{itemize} 
  	Uwagi: batTemp to temperatura zmierzone na kolejnych bateriach. Są to wartości 8 bitowe (uint8), pomiar w zakresie 0-127 stopni z krokiem co 0.5 stopnia. 
  	\item Ramka zawiera współrzędne GPS. Długość ramki: 8 znaków
  		\begin{itemize}
  		\item \arabic{enumi}
  		\item długośćGeo \& 0xFF;
  		\item (długośćGeo \& 0xFF00)>>8 ;
  		\item (długośćGeo \& 0xFF0000)>>16 ;
  		\item (długośćGeo \& 0xFF000000)>>24 ;
  		\item szerokośćGeo \& 0xFF;
  		\item (szerokośćGeo \& 0xFF00)>>8 ;
  		\item (szerokośćGeo \& 0xFF0000)>>16 ;
  		\item (szerokośćGeo \& 0xFF000000)>>24 ;
  		\end{itemize}
  		Uwagi: długość i szerokość to zmienne typu float zawierające informacje o stopniach (bez podziału na stopnie i minuty). Szerokość geograficzna (N i S), długość geograficzna (E i W).
  	\item Dane z IMU. Długość ramki: 12 znaków
  		\begin{itemize}
  		\item \arabic{enumi}
		\item przyspieszenieX \& 0xFF;
  		\item (przyspieszenieX \& 0xFF00)>>8 ;
  		\item przyspieszenieY \& 0xFF;
  		\item (przyspieszenieY \& 0xFF00)>>8 ;
  		\item przyspieszenieZ \& 0xFF;
  		\item (przyspieszenieZ \& 0xFF00)>>8 ;
  		\item żyroskopX \& 0xFF;
  		\item (żyroskopX \& 0xFF00)>>8 ;
  		\item żyroskopY \& 0xFF;
  		\item (żyroskopY \& 0xFF00)>>8 ;
  		\item żyroskopZ \& 0xFF;
  		\item (żyroskopZ \& 0xFF00)>>8 ;	
  		\end{itemize}
  		Uwagi: przyspieszenie i żyroskop to int16 zawierające odczyty z IMU, wymagają przeliczenia na wskazania przyspieszenia i prędkości kątowej.
  	\item Dane z Magnetometru. Długość ramki: 6 znaków
  		\begin{itemize}
  		\item \arabic{enumi}
  		\item magnetometrX \& 0xFF;
  		\item (magnetometrX \& 0xFF00)>>8 ;
  		\item magnetometrY \& 0xFF;
  		\item (magnetometrY \& 0xFF00)>>8 ;
  		\item magnetometrZ \& 0xFF;
  		\item (magnetometrZ \& 0xFF00)>>8 ;
  		\end{itemize}
  		Uwagi: magnetoemtr to int16 zawierający odczyty z Magnetometru.
  	\item Temperatura silników po prawej stronie. Długość ramki: 3 znaki
  		\begin{itemize}
  		\item \arabic{enumi}
  		\item tempSilnik1
  		\item tempSilnik2
  		\item tempSilnik3
		\end{itemize}  		
	Uwagi: tempSilnik to temperatura zmierzone na kolejnych silnikach. Są to wartości 8 bitowe (uint8), pomiar w zakresie 0-128 stopni z krokiem co 0.5 stopnia.  
	\item Temperatura silników po lewej stronie. Długość ramki: 3 znaki
  		\begin{itemize}
  		\item \arabic{enumi}
  		\item tempSilnik1
  		\item tempSilnik2
  		\item tempSilnik3
		\end{itemize}  		
	Uwagi: tempSilnik to temperatura zmierzone na kolejnych silnikach. Są to wartości 8 bitowe (uint8), pomiar w zakresie 0-128 stopni z krokiem co 0.5 stopnia.  
	\item Ograniczenie mocy silników. Ramka przesyłana do łazika. Długość ramki: 1 znak
		\begin{itemize}
		\item \arabic{enumi}
		\item Moc silnika
		\end{itemize}
	Uwagi: Moc silnika przyjmuje wartość od 1 do 20, gdzie 1 moc minimalna 20 pełna moc. 
	\item Status sterownika silników po prawej stronie. Długość ramki: 1 znak
		\begin{itemize}
		\item \arabic{enumi}
		\item Status sterownika silników prawy
		\end{itemize}
		Uwagi: Status przyjmuje następujące wartości:
		\begin{itemize}
		\item 1: Sterownik uruchomiony (start)
		\item 2: Sterownik zatrzymany (stop)
		\item 3: Brak zasilania silników 
		\item 4: Inny problem 
		\end{itemize}
	\item Status sterownika silników po lewej stronie. Długość ramki: 1 znak
		\begin{itemize}
		\item \arabic{enumi}
		\item Status sterownika silników lewy
		\end{itemize}
		Uwagi: patrz 114
	\item Status BMS'a
		\begin{itemize}
		\item 118
		\item Status sterownika silników lewy
		\end{itemize}
		Uwagi: Status przyjmuje następujące wartości:
		\begin{itemize}
		\item 1: OK
		\item 2: zbyt niskie napięcie
		\item 3: zbyt wysoka temp
		\item 4: zbyt wysoki prąd
		\end{itemize}
	\item Maksymalna prędkość
		\begin{itemize}
		\item 119
		\item Współczynnik prędkości maksymalnej 
		\end{itemize}
		Uwagi: Współczynnik prędkości maksymalnej przyjmuje wartości od 1 do 20, gdzie 1 bardzo wolno, 20 moc maksymalna.
	\item Ograniczenie prądu
		\begin{itemize}
		\item 120
		\item Współczynnik ograniczenia prądu
		\end{itemize}
		Uwagi: Współczynnik ograniczenia prądu przyjmuje wartości od 1 do 200. 
	\end{enumerate}
\subsubsection*{160-199 platforma jezdna}
\subsubsection*{200-239 platforma jezdna}	

\section{Konwersja danych}

Wartości przesłane za pomocą protokołów komunikacyjnych opisanych w poprzednim rozdziale wymagają konwersji, aby można było je wyświetlić i analizować w jednostkach SI. Poniższy rozdział zawiera niezbędne do tego wzory i przekształcenia. 
\subsection{currentValue}
W zależności od potrzeb w projekcie Strzyga zastosowano sterowniki silników wyposażone w dwa różne czujniki poboru prądu ACS714 o zakresie pomiaru \textpm 20A lub \textpm 50A. Charakteryzują się one różną czułością ($sens$). Dla czujnika 20A $sens$ = 100mV/A, a dla czujnika 50A $sens$ = 40mV/A. Czujniki 50A zastosowano w sterownikach silników napędowych, a 20A w sterownikach silników manipulatora. 
Dodatkowo pomiędzy wyjściem czujnika a wejściem mikrokontrolera zastosowano dzielnika napięcia z rezystorów $R_1 = 1k$ i $R_2 = 1.5k$. 
Przetwornik ADC na mikrokontrolerze pracuje z rozdzielczością 12 bitów i mierzy napięcie w zakresie od 0 do 3.3V.

Uwzględniając powyższe otrzymujemy zależność:
$$
\textrm{pobierany prąd[A]}=\frac{3.3*currentValue*(R_1+R_2)}{2^{12}*R_2*sens}
$$
\subsection{enkPrędkość}
Do pomiaru prędkości kół wykorzystywane są enkodery magnetyczne AMS AS5040B o rozdzielczości 12 bit pracujące w trybie kwadraturowym. Pomiar prędkości dokonywany jest poprzez pomiar czasu między wystąpieniem kolejnych zboczy narastających na sygnale z kanału A. Kanał B wykorzystywany jest do określenia kierunku obrotu. Enkoder umieszczony jest bezpośrednio na wale stąd nie ma konieczności uwzględniania w obliczeniach wpływu przekładni na prędkość. 
Wartość przesyłana jest zależna od preskalera timera $prescaler=250$, prescalera wejściowego $INprescaler=4$ oraz zegara taktującego timer $clock = 36MHz$.
Jeżeli enkPrędkość przyjmie wartość 0xFFFF oznacza to, że koło jest całkowicie zatrzymane. 
 %oraz wartości wyrażenia przez które jest dzielony czas między kolejnymi zboczami w celu uzyskania częstotliwości proporcjonalnej do prędkości kątowej $value = 4608000$.

Uwzględniając powyższe otrzymujemy zależność:
$$
\textrm{prędkość obrotowa [rpm]}=\frac{clock*INprescaler*60}{prescaler*enkPrędkość}
$$

\subsection{Vbat}
Odczyt napięcia na kolejnych celach baterii jest stosunkowo prosty, gdyż odczytujemy tutaj napięcie podzielone jedynie za pomocą rezystancyjnego dzielnika napięcia. Dla każdej z cel wykorzystano inny dzielnik. Dodatkowo należy pamiętać o sposobie przetwarzania napięcia opisanym w podrozdziale enkPrędkość

Odczyt napięcia sprowadza się do wzoru:
$$
\textrm{napięcie na baterii [V]}=\frac{3.3*Vbat*(R_1+R_2)}{2^{12}*R_1}
$$
Poniższa tabela zawiera wartości rezystorów $R_1$ i $R_2$ dla poszczególnych cel

\begin{tabular}{|c|c|c|}
\hline
 & $R_1$ & $R_2$ \\ \hline
 Vbat1 & 22k &6.8k \\ \hline
 Vbat2 & 12.4k & 22k \\ \hline
 Vbat3 & 10k & 33k \\ \hline
 Vbat4 & 10k & 47k \\ \hline
\end{tabular}

\subsection{batTemp}
Pomiar temperatury poszczególnych modułów bateryjnych dokonywany za pomocą termistorów NTC w układzie półmostkowym. Rezystor $R_1 = 10k$, rezystancja nominalna termistora $R_0 = 22k$. Konwersję na poziomie aplikacji opisuje wzór:
$$
\textrm{temperatura baterii }[^{\circ}C] = \frac{batTemp}{2} 
$$
Jednakże konwersja wykonywana na poziomie mikrokontrolera jest dużo bardziej złożona i opisana wzorem:
$$
\textrm{temperatura baterii }[^{\circ}C] = \frac{1}{A_1+B_1*ln\frac{R_1}{(\frac{4096}{ADC}-1)*R_0}}-273.15
$$
Gdzie $A_1$ oraz $B_1$ są parametrami zdefiniowanymi w dokumentacji termistora, a ADC jest bezpośrednim wynikiem pomiaru z przetwornika analogowo-cyfrowego.
\end{document}
